\title{Solution for Exersize 4}
\author{Johann Strama and Gorgin Torabi}
\date{\today}
\documentclass[12pt]{article}
\usepackage{mathtools}
\begin{document}
\maketitle

\begin{abstract}
    Solution for Exersize 8b
\end{abstract}

\section{Question 1b}

Set $p( \overrightarrow{x},\beta) = 0.5$ and take the $\beta^T = (-1.6,5.1)$ that was calculated after 1000 iterations and x is $\overrightarrow{x}^T = (1,x)$
\newline
\begin{math}
    \\
    p(\overrightarrow{x},\beta) = \frac{exp(\beta^T \overrightarrow{x})}{1+exp(\beta^T \overrightarrow{x})} \\
    0.5 =  \frac{exp((-1.6,5.1) \overrightarrow{x})}{1+exp((-1.6,5.1)  \overrightarrow{x})} \\
    0.5 = \frac{exp(-1.6+ 5.1x)}{1+exp(-1.6+ 5.1x)} \\
    \ln{0.5} = \ln{\frac{exp(-1.6+ 5.1x)}{1+exp(-1.6+ 5.1x)}} \\
    \ln{0.5} = \ln{exp(-1.6+ 5.1x)}- \ln{1+exp(-1.6+ 5.1x)} \\
    \ln{0.5} = -1.6+5.1x - \ln{1+exp(-1.6+ 5.1x)}\\
    \underline{\underline{x = 0.313725}}
\end{math}
\newline
Ab 0.31 mm Groe{ss}e der Sandkoerner ist es wahrscheinlicher Spinnen anzutreffen.
\end{document}
