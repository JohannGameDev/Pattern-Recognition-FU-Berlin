\title{Solution for Exersize 4}
\author{Johann Strama}
\date{\today}
\documentclass[12pt]{article}
\usepackage{mathtools}
\begin{document}
\maketitle

\begin{abstract}
Solution for Exersize 4
\end{abstract}

\section{Question 1}
\subsection{a)}
Mean  $\mu$ of givin points \newline
\begin{math}
    x : \frac{1+2+4+5}{4} = \frac{12}{4} = 3 \\
    y : \frac{-1+1+-1+1}{4} = \frac{0}{4} = 0 \\
\mu = \begin{pmatrix}
        3 \\
        0
    \end{pmatrix}
\end{math}
\subsection{b)}
The Covariancematrix C has the following form:
\begin{gather*}
    C = 
    \begin{pmatrix}
        covariance(x,x) & covariance(x,y) \\
        covariance(y,x) & covariance(y,y)
    \end{pmatrix}
    =
    \begin{pmatrix}
        variance(x) & covariance(x,y) \\
        covariance(y,x) & variance(y)
    \end{pmatrix}
    \\ \\
    covariance(x,y) = covariance(y,x) =  \frac{\displaystyle\sum_{i=0}^{n} (x_i - \bar{x}) * (y_i - \bar{y}) }{n}
    \\ \\
    covariance(x,y) = \frac{(1-3)*(-1-0) + (2-3)*(1-0) + (4-3)*(-1-0)+(5-3)*(1-0)}{4} \\ 
    = \frac{2}{4} = 2
    \\ \\
    variance(x) = \frac{\displaystyle\sum_{i=0}^{n} (x_i -\bar{x})^2}{n}
    \\ \\
    variance(x) = \frac{(1-3)^2 + (2-3)^2 +(4-3)^2 +(5-3)^2}{4-1} = \frac{ 4 + 1 + 1 + 4}{4}= \frac{10}{4}
    \\ \\
    variance(y) = \frac{(-1 -0)^2 + (1-0)^2 +(-1 -0)^2 + (1-0)^2}{4-1} = \frac{ 1+ +1 +1 +1}{4} = \frac{4}{4} = 1
    \\ \\
   C =  \begin{pmatrix}
       \frac{10}{4} & 2 \\
       2 &  1
    \end{pmatrix}
    =\begin{pmatrix}
        \frac{5}{2} & 2 \\
       2 &  1
    \end{pmatrix}
\end{gather*}
\subsection{c)}
\begin{gather*}
    det(C - I\lambda) = |C - I\lambda |= 0  \\
    \\
    \left|
    \begin{pmatrix}
        \frac{5}{2} & 2 \\
        2 & 1 
    \end{pmatrix}
    -
    \begin{pmatrix}
        1 & 0 \\
        0 & 1 
    \end{pmatrix}
    * \lambda \right| = 0 
\\
    \left|
    \begin{pmatrix}
        \frac{5}{2} & 2 \\
        2 & 1 
    \end{pmatrix}
    -
    \begin{pmatrix}
        \lambda & 0 \\
        0 & \lambda 
    \end{pmatrix}
    \right| = 0 
    \\
    \begin{vmatrix}
        \frac{5}{2} - \lambda & 2 \\
        2 & 1 -\lambda
    \end{vmatrix}
    = 0 \\
\end{gather*}
The formula for determinant is $$ 
\begin{vmatrix}
    a & b \\
    c & d 
\end{vmatrix}
= a * d - c * d $$
\begin{gather*}
    \begin{vmatrix}
        \frac{5}{2} - \lambda & 2 \\
        2 & 1 -\lambda
    \end{vmatrix}
    =(\frac{5}{2} - \lambda * 1 - \lambda) - (2 * 2) =\lambda^2 - \frac{5}{2}\lambda - \lambda + \frac{5}{2} - 4 =  \lambda^2 -\frac{7}{2}\lambda -\frac{3}{2} \\
    \text{p-q formula} \\
    \lambda_{1,2} =  -\frac{-\frac{7}{2}}{2} \pm \sqrt{\frac{-\frac{7}{2}}{2} - (-\frac{3}{2})} = \frac{7}{4} \pm \sqrt{\frac{49}{16}+ \frac{24}{16}} = \frac{7}{4} \pm \frac{\sqrt{73}}{4} \\
    \lambda_1 \approx \frac{7}{4} + 2 \approx \frac{15}{4} \\
    \lambda_2 \approx \frac{7}{4} - 2 \approx \frac{-8}{4}\approx \frac{-8}{4}\approx -2  
\end{gather*}







\subsection{d)}
Insert $\lambda_1$ in $(C - I\lambda_1) * x = 0 $
\begin{gather*}
    \left (
        \begin{pmatrix}
            \frac{5}{2} & 2 \\
            2 & 1 
    \end{pmatrix}
    -
   \begin{pmatrix}
       1 & 0 \\
            0 & 1 
    \end{pmatrix} 
    * \frac{15}{4} \right ) * x = 
    \left (
        \begin{pmatrix}
            \frac{5}{2} & 2 \\
            2 & 1 
    \end{pmatrix}
    -
   \begin{pmatrix}
       \frac{15}{4}  & 0 \\
       0 & \frac{15}{4} 
    \end{pmatrix} 
\right ) * x  \\
= \begin{pmatrix}
            \frac{5}{2} -\frac{15}{4} & 2 \\
            2 & 1 -\frac{15}{4} 
        \end{pmatrix} * x = 0 
\end{gather*}
Now  write generell formula
\begin{gather*}
 \begin{pmatrix}
            \frac{5}{2} -\frac{15}{4} & 2 \\
            2 & 1 -\frac{15}{4} 
        \end{pmatrix}  
        *
        \begin{pmatrix}
            x_1 \\
            x_2
        \end{pmatrix}
        =
        0
\end{gather*}
Make a system of linear equations
\begin{align}
    (\frac{5}{2} -\frac{15}{4})x_1 + 2x_2 = 0  \\
    2x_1 +(1 -\frac{15}{4})x_2 = 0 
\end{align}
\begin{align}
    -\frac{5}{4}x_1 + 2x_2 = 0  \\
    2x_1 -\frac{11}{4}x_2 = 0 
\end{align}
\begin{gather*}
    x_1 = \frac{11}{8}\\
    x_2 = 1 \\
    \text{the first eigenvector is } x = 
    \begin{pmatrix}
        \frac{11}{8} \\
        1 
   \end{pmatrix}
\end{gather*}
And now the second eigenvector!
Insert $\lambda_2$ in $(C - I\lambda_2) * x = 0 $
\begin{gather*}
    \left (
        \begin{pmatrix}
            \frac{5}{2} & 2 \\
            2 & 1 
    \end{pmatrix}
    -
   \begin{pmatrix}
       1 & 0 \\
            0 & 1 
    \end{pmatrix} 
    * -2  \right ) * x = 
    \left (
        \begin{pmatrix}
            \frac{5}{2} & 2 \\
            2 & 1 
    \end{pmatrix}
    -
   \begin{pmatrix}
       -2  & 0 \\
       0 & -2 
    \end{pmatrix} 
\right ) * x  \\
= \begin{pmatrix}
    \frac{5}{2} + 2  & 2 \\
    2 & 1 +2  
        \end{pmatrix} * x = 0 
\end{gather*}
Now  write generell formula
\begin{gather*}
 \begin{pmatrix}
     \frac{5}{2} +2 & 2 \\
     2 & 1 +2 
        \end{pmatrix}  
        *
        \begin{pmatrix}
            x_1 \\
            x_2
        \end{pmatrix}
        =
        0
\end{gather*}
Make a system of linear equations
\begin{align}
    (\frac{5}{2} + 2 + 2x_2 = 0  \\
    2x_1 +1 + 2 = 0 
\end{align}
\begin{align}
    \frac{9}{2}x_1 + 2x_2 = 0  \\
    2x_1 + 3x_2  = 0 
\end{align}
\begin{gather*}
    x_1 = -\frac{5}{8}\\
    x_2 = 1 \\
    \text{the second eigenvector is } x = 
    \begin{pmatrix}
        - \frac{5}{8}\\
        1
   \end{pmatrix}
\end{gather*}

\subsection{e)}

\end{document}
